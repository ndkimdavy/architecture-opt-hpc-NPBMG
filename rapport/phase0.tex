\begin{abstract}
Ce travail présente une analyse de performance du benchmark Multi-Grid (MG) issu de la collection de kernels NAS Parallel Benchmarks (NPB), un ensemble d'outils développés pour reproduire les caractéristiques de calcul et de transfert de données des applications de mécanique des fluides (CFD) à grande échelle.
La particularité de ce benchmark réside dans sa spécification "pencil and paper" \cite{nasbenchmarks1994}: il définit l'algorithme de manière théorique pour garantir une portabilité totale, laissant ainsi aux développeurs la liberté d'optimiser l'implémentation selon l'architecture cible.
L’objectif est d’évaluer les différentes approches de parallélisation, notamment l’approche à mémoire partagée avec OpenMP, tout en analysant l’efficacité des différentes toolchains de compilation.

L’analyse s’appuie sur l’outil MAQAO (Modular Assembly Quality Analyzer and Optimizer) \cite{maqao}, un framework d'analyse et d'optimisation des performances qui permet de mettre en lumière l'influence directe des choix d'implémentation sur l'exploitation des ressources matérielles.
Les résultats indiquent que le choix du paradigme de langage (Fortran ou C++) et du compilateur influence significativement les performances observées, en raison des différences de génération de code, de vectorisation et d’exploitation du parallélisme.
L’étude examine également la perte d’efficacité parallèle (scalability gap) ainsi que l’impact des coûts de communication lors de la montée en charge.
En s'appuyant sur des diagnostics de qualité de code, l’analyse identifie les principaux goulots d’étranglement matériels (bottlenecks) et propose des pistes d'optimisation ciblées pour améliorer la vectorisation au sein des routines critiques.
\end{abstract}